% Metódy inžinierskej práce

\documentclass[10pt,twoside,slovak,a4paper]{coursepaper}

\usepackage[slovak]{babel}
%\usepackage[T1]{fontenc}
\usepackage[IL2]{fontenc} % lepšia sadzba písmena Ľ než v T1
\usepackage[utf8]{inputenc}
\usepackage{graphicx}
\usepackage{url} % príkaz \url na formátovanie URL
\usepackage{hyperref} % odkazy v texte budú aktívne (pri niektorých triedach dokumentov spôsobuje posun textu)

\usepackage{cite}
%\usepackage{times}

\pagestyle{headings}

\title{Gamifikácia a jej aplikácia v zlepšovaní interakčných schopností detí s autizmom\thanks{Semestrálny projekt v predmete Metódy inžinierskej práce, ak. rok 2015/16, vedenie: Meno Priezvisko}} % meno a priezvisko vyučujúceho na cvičeniach

\author{Meno Priezvisko\\[2pt]
	{\small Slovenská technická univerzita v Bratislave}\\
	{\small Fakulta informatiky a informačných technológií}\\
	{\small \texttt{...@stuba.sk}}
	}

\date{\small 30. september 2015} % upravte



\begin{document}

\maketitle

\begin{abstract}
Gamifikácia je pojem, respektíve technika, ktorá je relatívne nová, no jej využitie a pozitívny vplyv môžu predstavovať prevrat v mnohých oblastiach. Doposiaľ sa gamifikácia využívala najmä v pracovnom prostredí, kde mala za úlohu motivovať zamestnancov, zvýšiť ich profesionalitu a vzdelanie. Samozrejme, gamifikácia nemá využitie iba v pracovnom prostredí, ale môže byť cielená aj na určité skupiny, ktorým môže mimoriadne pomôcť. 
V tomto článku by som sa rád zameral práve na jednu skupinu, a to deti s autizmom, pre ktoré by gamifikácia mohla byť revolučná v zmysle pomoci pri zlepšení ich interakčných schopností, s ktorými majú nielen deti no i dospelí ľudia s autizmom problém. Nakoľko nie je mnoho výskumov, ktoré by inkorporovali deti s autizmom v gamifikácii, tak verím, že tento článok pomôže v priblížení danej problematiky širšej verejnosti, a tak odštartuje nové a detailnejšie výskumy, ktoré pomôžu zjednodušiť život deťom no i ľudom s touto vývojovou poruchou. 
\end{abstract}



\section{Úvod}

Autizmus je všeobecne známa vývojová porucha, s ktorou sa narodí približne jedno dieťa zo sto. Aj napriek tomu, že autizmus ako vývojová porucha je komplexne preskúmaná a aj v súčasnosti sa jej venuje veľké množstvo výskumných organizácií a tímov, jeden integrálny aspekt doposiaľ nezískal dostatočnú pozornosť. Je to práve problematika moderného prístupu k zlepšovaniu interakčných schopností detí s autizmom. Práve táto myšlienka bude predstavovať gro tejto témy. V tomto článku sa budem venovať priblíženiu vývojovej poruchy autizmu, jeho rozdeleniu a dištinkcií medzi typmi autizmu, ďalej pojmu gamifikácia, jej využitiu v modernej spoločnosti a v neposlednom rade integrácii detí s autizmom v gamifikačnom procese s cieľom zlepšenia ich interakčných schopností. Na záver rozoberiem prínosy gamifikácie v spomenutom kontexte a priblížim možné napredovanie v tomto smere.   
Uveďte explicitne štruktúru článku. Tu je nejaký príklad.
Základný problém, ktorý bol naznačený v úvode, je podrobnejšie vysvetlený v časti~\ref{nejaka}.
Dôležité súvislosti sú uvedené v častiach~\ref{dolezita} a~\ref{dolezitejsia}.
Záverečné poznámky prináša časť~\ref{zaver}.



\section{Predstavenie problematiky u detí s autizmom} 
	Autizmus sa klasifikuje ako pervazívna porucha vývinu, čo znamená, že je hlboko prenikajúca a to implikuje, že vývoj bol prerušený, respektíve narušený vo všetkých kľúčových oblastiach. Ľudia, ktorí touto vývojovou poruchou trpia sú často každodenne znevýhodnení v rôznych oblastiach, nakoľko autizmus postihuje do určitej miery všetky kognitívne no často i motorické funkcie tela. To isté platí aj pre deti s autizmom, na ktoré tieto symptómy vplývajú o to intenzívnejšie, pretože tieto deti nemali ešte dostatok času prispôsobiť sa takýmto náročným podmienkam. Deti, ktoré sa nachádzajú na spektre autizmu často potrebujú dennodennú asistenciu a to najmä v situáciách, v ktorých je potrebná verbálna komunikácia či akákoľvek interakcia s okolím. Tieto deti nielenže často nedokážu vyjadriť svoje názory, alebo pocity, ale taktiež majú problém s nadviazaním nových priateľstiev či známostí. O to komplikovanejšie je to z dôvodu, že sa pohybujú v okruhoch svojich rovesníkov, ktorí často nevedia čo je autizmus a deťom s autizmom sa vyhýbajú, nakoľko nerozumejú ich správaniu a reakciám.  



\subsection{Autizmus}
	Prejavy autizmu sa líšia nie len intenzitou no i kvantitou či spôsobom. Medzi základné prejavy patria uhýbanie sa očnému kontaktu, oneskorená schopnosť rozprávať, repetičné správanie, alebo aj seba poškodzovanie. Nie všetci ľudia s autizmom sa budú správať rovnako, no niektoré prejavy spájané s autizmom budú mať spoločné. Práve z toho dôvodu je klasifikácia do istej miery variabilná a nie úplne istá, i keď historicky sa autizmus vždy delil do pod kategórií. V psychiatrii sa autizmus považuje sa neurovývojovú poruchu no veľa ľudí s autizmom považujú autizmus za súčasť neurodiverzity, čo znamená rozmanitosť v spôsobe myslenia, správania a skúseností. Táto nezhoda v snahe o konkrétne označenie autizmu viedla k sporom medzi výskumníkmi, psychiatrami a ľuďmi s autizmom.    

\begin{figure*}[tbh]
\centering
\includegraphics[scale=0.5]{vyskyt_autizmu.pdf}
% Aj text môže byť prezentovaný ako obrázok. Stane sa z neho označný plávajúci objekt. Po vytvorení diagramu zrušte znak \texttt{\%} pred príkazom \verb|\includegraphics| označte tento riadok ako komentár (tiež pomocou znaku \texttt{\%}).
\caption{Výskyt autizmu medzi deťmi je 1 ku 100}
\label{f:rozhod}
\end{figure*}



\subsection{Príčiny autizmu}
	Čo spôsobuje autizmus je doposiaľ diskutované medzi výskumníkmi a vedcami, no je známe, že autizmus je často dedičný a spôsobený genetickými abnormalitami. Je predpokladané, že pravdepodobne neexistuje jeden špecifický dôvod na vznik autizmu, i keď vedci dlho uprednostňovali možnosť existencie spoločnej príčiny autizmu u detí. Aj keď nie je konkrétne definovaný jednoznačný dôvod, existujú rizikové faktory, ako napríklad genetika, predčasné narodenie či životný štýl rodičov počas tehotenstva. V súčasnosti neexistuje liečivo na autizmus, no je odporúčaná skorá intervencia z dôvodu rýchlejšej adaptácie dieťaťa na nové spôsoby komunikácie ako napríklad neverbálna komunikácia. 


\subsection{Členenie autizmu}
Ako bolo spomenuté autizmus sa klasifikuje medzi pervazívne vývojové poruchy, tiež známe pod označením F84. Ďalej sa autizmus môže členiť na základe každodenných potrieb človeka, ktorý sa nachádza na spektre autizmu, nakoľko sa tieto potreby individuálne líšia. Niektorí potrebujú stálu starostlivosť, pomoc s komunikáciu či vykonávaním fundamentálnych každodenných funkcií, iní nepotrebujú takmer žiadnu pomoc a sú schopní viesť svoj život takmer bez akýchkoľvek obmedzení. Hlavné členenie autizmu, ktoré je všeobecne akceptované:  

\begin{itemize}
\item Dezintegračná porucha v detstve
\item Detský autizmus
\item Rettov syndróm
\item Aspergerov syndróm 
\item Atypický autizmus
\end{itemize}


\section{Gamifikácia}

Základným problémom je teda\ldots{} Najprv sa pozrieme na nejaké vysvetlenie (časť~\ref{ina:nejake}), a potom na ešte nejaké (časť~\ref{ina:nejake}).\footnote{Niekedy môžete potrebovať aj poznámku pod čiarou.}

Môže sa zdať, že problém vlastne nejestvuje\cite{Coplien:MPD}, ale bolo dokázané, že to tak nie je~\cite{Czarnecki:Staged, Czarnecki:Progress}. Napriek tomu, aj dnes na webe narazíme na všelijaké pochybné názory\cite{PLP-Framework}. Dôležité veci možno \emph{zdôrazniť kurzívou}.


\subsection{Aplikácia gamifikácie} \label{Aplikácie gamifikácie}

Niekedy treba uviesť zoznam:

\begin{itemize}
\item jedna vec
\item druhá vec
	\begin{itemize}
	\item x
	\item y
	\end{itemize}
\end{itemize}

Ten istý zoznam, len číslovaný:

\begin{enumerate}
\item jedna vec
\item druhá vec
	\begin{enumerate}
	\item x
	\item y
	\end{enumerate}
\end{enumerate}


\subsection{Ešte nejaké vysvetlenie} \label{ina:este}

\paragraph{Veľmi dôležitá poznámka.}
Niekedy je potrebné nadpisom označiť odsek. Text pokračuje hneď za nadpisom.



\section{Integrácia detí s autizmom v gamifikačnom procese} \label{Integrácia detí s autizmom v gamifikačnom procese}




\section{Vízie do budúcnosti} \label{Vízie do budúcnosti}




\section{Záver} \label{zaver} % prípadne iný variant názvu



\section{Zaver2} \label{zaver 2}




%\acknowledgement{Ak niekomu chcete poďakovať\ldots}


% týmto sa generuje zoznam literatúry z obsahu súboru literatura.bib podľa toho, na čo sa v článku odkazujete
\bibliography{literatura}
\bibliographystyle{abbrv} % prípadne alpha, abbrv alebo hociktorý iný
\end{document}
